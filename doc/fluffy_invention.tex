\documentclass[a4paper, 11pt]{article}
\usepackage{amsmath}
\usepackage{amssymb}
\usepackage{indentfirst}

\author{One \& Two}
\title{Surface minimale en dimension finie}

\begin{document}
\maketitle
\newpage

\section{Probl\`eme de surface minimale en dimension infinie}
Soit $\Omega$ un sous-ensemble born\'e de $\mathbb{R}^2$ \`a fronti\`ere
 lipschitzienne. Soit $g\in\mathcal{C}^0(\partial\Omega)$. Le probl\`eme de
 surface minimale est le suivant:
\begin{equation}
\begin{cases}
\min\limits_{u\in H^1(\Omega)}\int\limits_\Omega\sqrt{1+|\nabla u|^2}\,
\mathrm{d}\lambda \\
u_{|\partial\Omega}=g
\end{cases}
\end{equation}
On pose:
$$
\forall x\in\Omega, \forall u\in H^1(\Omega), L(x, u(x), \nabla u(x))=
\sqrt{1+|\nabla u(x)|^2}
$$
Et on note:
$$
\forall u\in H^1(\Omega), J(u)=\int\limits_\Omega L(x, u(x), \nabla u(x))
\,\mathrm{d}\lambda
$$
Le probl\`eme de surface minimale est le suivant:
\begin{equation}
\begin{cases}
\min\limits_{u\in H^1(\Omega)}J(u) \\
u_{|\partial\Omega}=g
\end{cases}
\end{equation}
Si la fonctionnelle $J$ admet un minimum $u$ alors pour tous $v\in H^1(\Omega)$,
 la diff\'erentielle de $J$ verifie:
\begin{align*}
\langle dJ(u), v\rangle &= \int\limits_\Omega\langle dL(x, u(x), \nabla u(x)),
v(x)\rangle\,\mathrm{d}\lambda \\
&= \int\limits_\Omega\frac{\langle \nabla u(x), \nabla v(x)\rangle}{L(x, u(x),
 \nabla u(x))}\,\mathrm{d}\lambda
\end{align*}
Et:
$$
\langle dJ(u), v\rangle = 0
$$
\newpage
\section{Probl\`eme de surface minimale en dimension finie}
Pour r\'esoudre le probl\`eme de surface minimal en dimension finie, on utilise
 la m\'ethode des \'el\'ements finies.

Soit $h>0$. Soient $\mathcal{T}$ une triangulation du domaine $\Omega$,
 $\mathcal{T}_{\partial\Omega}\subset\mathcal{T}$ l'ensemble des triangles avec
 au moins un ar\^ete faisant partie du bord de $\Omega$, $V_h$
 l'ensemble des fonctions P1-Lagrange.

On r\'esout alors le probl\`eme suivant:
\begin{equation}
\begin{cases}
\min\limits_{u_h\in V_h} J(u_h) \\
u_{|\partial\Omega}=g
\end{cases}
\end{equation}

On indice l'ensemble des sommets par un ensemble $I$ et on indice l'ensemble des
 sommets sur le bord par $I_{\partial\Omega}$.
Si $u_h\in V_h$ est une solution alors pour tout $v_h\in V_h$,
$$
\langle dJ(u_h), v_h\rangle = 0
$$
Cependant $dJ$ n'est pas lin\'eaire en $u_h$, on d\'efinit alors la suite
 $(u_n)$ de $V_h$ et on substitue $dJ$ par:
$$
\forall (u_{n+1}, v)\in (V_h)^2, \kappa(u_{n+1}, v)=\int\limits_\Omega
\frac{\langle\nabla u_{n+1}(x), \nabla v(x)\rangle}{L(x, u_n(x), \nabla u_n(x))}
\,\mathrm{d}\lambda
$$

$$
\dots
$$
On r\'esout alors le syst\`eme suivant:
$$
\forall i\in I, \sum_{j\in I}(u_n)_j\kappa(\phi_j, \phi_i)=0
$$
soit,
$$
\forall i\in I, \sum\limits_{j\in I\setminus I_{\partial\Omega}}
(u_n)_j\kappa(\phi_j, \phi_i)=-\sum\limits_{j\in I_{\partial\Omega}}
g_j\kappa(\phi_j, \phi_i)
$$
\end{document}

